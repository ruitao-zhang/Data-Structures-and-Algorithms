\documentclass[a4paper]{article}
\usepackage[affil-it]{authblk}
\usepackage[backend=bibtex,style=numeric]{biblatex}
\usepackage{listings}
\usepackage{indentfirst}
\setlength{\parindent}{2em}
\usepackage{xcolor}
\usepackage{float}
\usepackage{amsfonts}
\usepackage{tikz}
\usepackage{xeCJK}
\usepackage{amsmath,amssymb}
\usepackage{graphicx}
\usepackage{booktabs}
\setlength{\parindent}{2em}
\lstset{
  language= C++,                
  basicstyle=\ttfamily\small,   
  keywordstyle=\color{blue},    
  stringstyle=\color{red},      
  commentstyle=\color{red},   
  numbers=left,                 
  numberstyle=\tiny,           
  stepnumber=1,                 
  tabsize=4,                    
  showspaces=false,             
  showstringspaces=false        
}
\usepackage{geometry}
\geometry{margin=1.5cm, vmargin={0pt,1cm}}
\setlength{\topmargin}{-1cm}
\setlength{\paperheight}{29.7cm}
\setlength{\textheight}{25.3cm}

\addbibresource{citation.bib}
\linespread{1.5}

\begin{document}
\title{报告}

\author{张睿涛 Zhang Ruitao 3210105252
  \thanks{Electronic address: \texttt{3210105252@zju.edu.cn}}}
\affil{(Information and Computing Science 2101), Zhejiang University }

\date{Due time: \today}

\maketitle

\section*{A}
\begin{figure}[H]
    \centering
    \includegraphics[width=0.8\textwidth]{figure/1.1.png}
    \caption{}
    \label{fig:example}
\end{figure}

\begin{figure}[H]
    \centering
    \includegraphics[width=0.8\textwidth]{figure/1.2.png}
    \caption{}
    \label{fig:example}
\end{figure}
\begin{figure}[H]
    \centering
    \includegraphics[width=0.8\textwidth]{figure/1.3.png}
    \caption{}
    \label{fig:example}
\end{figure}
\begin{figure}[H]
    \centering
    \includegraphics[width=0.8\textwidth]{figure/1.4.png}
    \caption{}
    \label{fig:example}
\end{figure}
\begin{figure}[H]
    \centering
    \includegraphics[width=0.8\textwidth]{figure/1.5.png}
    \caption{}
    \label{fig:example}
\end{figure}
\begin{figure}[H]
    \centering
    \includegraphics[width=0.8\textwidth]{figure/1.6.png}
    \caption{}
    \label{fig:example}
\end{figure}
\section*{B}
\begin{figure}[H]
    \centering
    \includegraphics[width=0.8\textwidth]{figure/1.6.png}
    \caption{}
    \label{fig:example}
\end{figure}
\begin{figure}[H]
    \centering
    \includegraphics[width=0.8\textwidth]{figure/1.6.png}
    \caption{}
    \label{fig:example}
\end{figure}
\section*{C}
\begin{figure}[H]
    \centering
    \includegraphics[width=0.8\textwidth]{figure/3.png}
    \caption{}
    \label{fig:example}
\end{figure}
\section*{D}
三次B样条插值在指定点的误差:\\
E_S(-3.5) = -0.07241896540758129\\
E_S(-3) = -0.08499955496859522\\
E_S(-0.5) = -0.7648317286296568\\
E_S(0) = -0.7813461709627464\\
E_S(0.5) = -0.7648317286296568\\
E_S(3) = -0.08499955496859522\\
E_S(3.5) = -0.07241896540758129\\

二次B样条插值在指定点的误差:\\
E_S(-3.5) = -0.02831996281550777\\
E_S(-3) = -0.06599384706312209\\
E_S(-0.5) = -0.20653891247038558\\
E_S(0) = -0.7967291109128755\\
E_S(0.5) = -0.5803775311811162\\
E_S(3) = -0.08164219218754697\\
E_S(3.5) = -0.04959340070201645\\

\begin{enumerate}
    \item \textbf{为什么有些误差接近机器精度?}当插值节点恰好位于插值点时,误差往往非常小。特别是在节点附近,B样条插值可以精确地逼近原函数,因为这些点本身是插值点。又因为数值计算的精度受到计算机浮点数表示的限制,因此误差会收敛到一个接近机器精度的小数值。
    \item \textbf{哪种B样条插值更加准确?}三次比二次更加精确。这是因为三次插值具有更高的平滑度,尤其是在函数的导数(斜率)处。从误差的大小看,三次的误差一般要比二次小,因此更适合用于较高平滑度要求
\end{enumerate}


\section*{E}
\begin{figure}[H]
    \centering
    \includegraphics[width=0.8\textwidth]{figure/5.1.png}
    \caption{}
    \label{fig:example}
\end{figure}
\begin{figure}[H]
    \centering
    \includegraphics[width=0.8\textwidth]{figure/5.2.png}
    \caption{}
    \label{fig:example}
\end{figure}
\section*{F}
\begin{figure}[H]
    \centering
    \includegraphics[width=0.8\textwidth]{figure/6.1.png}
    \caption{}
    \label{fig:example}
\end{figure}
分差表:
['(0 - x)_+^2' '(1 - 0)' '(2 - 0)']
['(1 - x)_+^2' '(2 - 1)' 0]
['(2 - x)_+^2' 0 0]
\end{document}
