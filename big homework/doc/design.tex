\documentclass[a4paper]{article}
\usepackage[affil-it]{authblk}
\usepackage[backend=bibtex,style=numeric]{biblatex}
\usepackage{listings}
\usepackage{indentfirst}
\setlength{\parindent}{2em}
\usepackage{xcolor}
\usepackage{float}
\usepackage{amsfonts}
\usepackage{tikz}
\usepackage{xeCJK}
\usepackage{amsmath,amssymb}
\usepackage{graphicx}
\usepackage{booktabs}
\setlength{\parindent}{2em}
\lstset{
  language= C++,                
  basicstyle=\ttfamily\small,   
  keywordstyle=\color{blue},    
  stringstyle=\color{red},      
  commentstyle=\color{red},   
  numbers=left,                 
  numberstyle=\tiny,           
  stepnumber=1,                 
  tabsize=4,                    
  showspaces=false,             
  showstringspaces=false        
}
\usepackage{geometry}
\geometry{margin=1.5cm, vmargin={0pt,1cm}}
\setlength{\topmargin}{-1cm}
\setlength{\paperheight}{29.7cm}
\setlength{\textheight}{25.3cm}

\addbibresource{citation.bib}
\linespread{1.5}

\begin{document}
\title{设计与数学说明}

\author{张睿涛 Zhang Ruitao 3210105252
  \thanks{Electronic address: \texttt{3210105252@zju.edu.cn}}}
\affil{(Information and Computing Science 2101), Zhejiang University }

\date{Due time: \today}

\maketitle

\section{BSplineBasis.h}
\subsection{结构说明}
\begin{itemize}
    \item \textbf{成员变量:}
    \begin{itemize}
        \item \texttt{k}:存储节点序列的 \texttt{std::vector<double>} 类型容器。
        \item \texttt{degree}:表示B-Spline的阶数。
    \end{itemize}
    \item \textbf{成员函数:}
    \begin{itemize}
        \item \texttt{BSplineBasis(const std::vector<double>& k, int degree)}:初始化节点序列和B-Spline的阶数,并确保节点数至少为 \texttt{degree + 2}。
        \item \texttt{f0(int i, int n, double t)}:递归计算B-Spline基函数的值。
        \item \texttt{getKnots()}:返回节点序列。
        \item \texttt{getDegree()}:返回B-Spline的阶数。
    \end{itemize}
\end{itemize}

\subsection{设计思路}

\subsection{递归}

- 对于0阶基函数 \( B_i^0(t) \),它是一个分段函数,在每个区间 \([k[i], k[i+1])\) 上取值1,其它地方取值0。
- 对于高阶基函数 \( B_i^n(t) \),其计算依赖于前一阶的基函数。递归的形式是:
\[
B_i^n(t) = \frac{t - \text{k}[i]}{\text{k}[i+n] - \text{k}[i]} B_i^{n-1}(t) + \frac{\text{k}[i+n+1] - t}{\text{k}[i+n+1] - \text{k}[i+1]} B_{i+1}^{n-1}(t)
\]

\subsection{检查}
f0
构造函数通过检查输入的节点序列长度是否满足至少为 \texttt{degree + 2} 个节点来确保节点的有效性。

\subsection{类的功能接口}

\subsubsection{构造函数}

\[
\texttt{BSplineBasis(const std::vector<double>& k, int degree)}
\]
初始化B-Spline基函数类,接受节点序列和B-Spline阶数。构造函数内部检查节点的数量是否满足B-Spline阶数的要求。

\subsubsection{基函数计算函数}

\[
\texttt{f0(int i, int n, double t)}
\]
该函数递归计算并返回B-Spline基函数 \( B_i^n(t) \) 的值。首先处理0阶基函数,然后递归计算更高阶的基函数。

\section{BSplineCurve.h}
\subsection{结构说明}
\begin{itemize}
    \item \textbf{成员变量:}
    \begin{itemize}
        \item \texttt{basis}:类型为 \texttt{BSplineBasis} 的对象,用于存储B-Spline基函数对象。这些基函数是计算B-Spline曲线值所必需的。
        \item \texttt{c}:类型为 \texttt{std::vector<double>} 的容器,存储B-Spline的权重。
    \end{itemize}
    \item \textbf{成员函数:}
    \begin{itemize}
        \item 构造函数 \texttt{BSplineCurve(const BSplineBasis\& basis, const std::vector<double>\& c)}:初始化B-Spline基函数和权重,并检查权重数量是否与B-Spline基函数匹配。
        \item \texttt{f1(double t)}:计算并返回B-Spline曲线在参数 \( t \) 下的值 \( S(t) \)。
    \end{itemize}
\end{itemize}

\subsection{设计思路}

\subsubsection{B-Spline曲线的构建}
\begin{enumerate}
    \item \textbf{初始化:} 在构造函数中,传入了一个B-Spline基函数对象 \texttt{basis} 和权重 \texttt{c}。且B-Spline的节点数应为权重数加上基函数阶数加1:
    \[
    \text{c} + \text{B-Spline阶数} + 1 = \text{节点数}
    \]
    如果不匹配,则非法。
    \item \textbf{曲线计算:} 在 \texttt{f1} 函数中,B-Spline曲线在给定参数 \( t \) 下的值 \( S(t) \) 通过所有点与其相应基函数的加权和计算:
    \[
    S(t) = \sum_{i} a_i B_i^n(t)
    \]
\end{enumerate}

\section{类的功能接口}

\subsection{构造函数}

\[
\texttt{BSplineCurve(const BSplineBasis\& basis, const std::vector<double>\& c)}
\]
该函数初始化B-Spline基函数对象和权重,并确保其相匹配。

\subsection{曲线计算函数}

\[
\texttt{f1(double t)}
\]
该函数计算并返回B-Spline曲线在给定参数 \( t \) 处的值。它通过遍历所有控制点,将控制点的权重与相应的B-Spline基函数的值相乘,然后加权求和得到曲线值:
\[
S(t) = \sum_{i} a_i B_i^n(t)
\]
其中 \( a_i \) 是第 \( i \) 个点,\( B_i^n(t) \) 是第 \( i \) 个基函数在参数 \( t \) 处的值。

\section{CubicSpline.h}
\subsection{类的结构说明}
\subsubsection{成员变量}
\begin{itemize}
    \item \( x \):存储样本点的横坐标。
    \item \( y \):存储样本点的纵坐标。
    \item \( a, b, c, d \):存储三次样条的系数,分别表示每个区间的三次多项式的常数项、一次项、二次项和三次项系数。
\end{itemize}

\subsection{成员函数}
\begin{itemize}
    \item \texttt{CubicSpline(const std::vector<double>& x, const std::vector<double>& y)}:构造函数,用于初始化并进行输入数据的验证。
    \item \texttt{f3(double xVal)}:根据给定的 \( x \) 值计算插值结果。
    \item \texttt{g0()}:计算三次样条的系数 \( a \), \( b \), \( c \), \( d \)。
    \item \texttt{g1(double xVal)}:找出给定 \( x \) 值所在的区间。
\end{itemize}

\section{设计思路}

\begin{itemize}
    \item \textbf{初始化和验证:} 初始化时检查 \( x \) 和 \( y \) 的有效性,保证 \( x \) 必须严格递增,且 \( x \) 和 \( y \) 的大小相等。
    \item \textbf{计算样条系数:} 通过三次样条插值法计算每个区间的四个系数 \( a \), \( b \), \( c \), \( d \)。
    \item \textbf{插值计算:} 给定 \( x \) 值,根据对应的三次多项式系数计算插值值。
\end{itemize}

\subsection{类的功能接口}
\subsection{构造函数}
\begin{verbatim}
CubicSpline(const std::vector<double>& x, const std::vector<double>& y)
\end{verbatim}
\begin{itemize}
    \item \textbf{功能:} 初始化三次样条插值类的对象,并检查输入数据的有效性
    \item \textbf{参数:} 
    \begin{itemize}
        \item x:一个包含横坐标的向量。
        \item y:一个包含纵坐标的向量。
    \end{itemize}
\end{itemize}

\subsection{插值函数}

\begin{verbatim}
double f3(double xval) const
\end{verbatim}

\begin{itemize}
    \item \textbf{功能:} 根据给定的 \( xval \) 计算在三次样条曲线上的插值结果。
    \item \textbf{返回值:} 给定 \( xval \) 对应的插值值。
\end{itemize}

\subsection{计算系数}

\begin{verbatim}
void g0()
\end{verbatim}

\begin{itemize}
    \item \textbf{功能:} 计算三次样条的四个系数 \( a \), \( b \), \( c \), \( d \),这四个系数用于表示每个区间内的三次多项式。
\end{itemize}

\subsection{寻找区间}

\begin{verbatim}
size_t g1(double xval) const
\end{verbatim}

\begin{itemize}
    \item \textbf{功能:} 给定 \( xval \),通过二分查找确定其所属的区间,并返回该区间的索引。
    \item \textbf{返回值:} \( xval \) 所在区间的索引。
\end{itemize}

\section{pp.h}
\subsection{结构说明}
\subsubsection{成员变量}

\begin{itemize}
    \item \texttt{c}:一个二维向量,每个子向量代表一个区间的多项式系数。每个子向量对应的多项式系数依次为常数项、一次项、二次项等。
    \item \texttt{b}:一个向量,存储多项式的分段点,为每个区间的端点。长度为 \(n+1\) 的向量表示 \(n\) 个区间。
\end{itemize}

\subsection{成员函数}

\begin{itemize}
    \item \texttt{PP}:构造函数,初始化分段点和每个区间的多项式系数。。
    \item \texttt{f4(double x)}:给定一个 \(x\) 值,计算并返回分段多项式的值。它遍历所有区间,找到包含 \(x\) 的区间,并使用对应的多项式系数计算值。
\end{itemize}

\subsection{设计思路}

\begin{itemize}
    \item 
    输入:构造函数接收两个参数:每个区间的多项式系数和分段点。函数 \texttt{f4} 接收一个 \(x\) 值,计算并返回在分段多项式上的插值结果。
    
    \item 输出:返回 \(x\) 对应的多项式值。

\end{itemize}

\subsection{类的功能接口}

\subsection{构造函数PP}

\begin{verbatim}
PP(const std::vector<std::vector<double>>& c, const std::vector<double>& b)
\end{verbatim}

\begin{itemize}
    \item \textbf{功能:} 初始化分段多项式对象。
    \item \textbf{参数:} 
    \begin{itemize}
        \item coefficients:一个二维向量,每个子向量包含一个区间的多项式系数。
        \item breakpoints:一个向量,包含分段点,表示每个区间的端点。
    \end{itemize}
\end{itemize}

\subsection{评估函数f4}

\begin{verbatim}
double f4(double x) const
\end{verbatim}

\begin{itemize}
    \item \textbf{功能:} 估计给定点 \( x \) 在分段多项式上的值。
    \item \textbf{返回值:} 对应 \( x \) 的多项式值。
\end{itemize}


\end{document}

