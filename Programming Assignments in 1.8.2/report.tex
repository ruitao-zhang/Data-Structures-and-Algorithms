\documentclass[a4paper]{article}
\usepackage[affil-it]{authblk}
\usepackage[backend=bibtex,style=numeric]{biblatex}
\usepackage{listings}
\usepackage{xcolor}
\usepackage{xeCJK}
\usepackage{amsmath}
\lstset{
  language=C++,                
  basicstyle=\ttfamily\small,   
  keywordstyle=\color{blue},    
  stringstyle=\color{green},      
  commentstyle=\color{red},   
  numbers=left,                 
  numberstyle=\tiny,           
  stepnumber=1,                 
  tabsize=4,                    
  showspaces=false,             
  showstringspaces=false        
}
\usepackage{geometry}
\geometry{margin=1.5cm, vmargin={0pt,1cm}}
\setlength{\topmargin}{-1cm}
\setlength{\paperheight}{29.7cm}
\setlength{\textheight}{25.3cm}

\addbibresource{citation.bib}

\begin{document}
% =================================================
\title{Programming Assignments in 1.8.2}

\author{Ruitao Zhang 3210105252
  \thanks{Electronic address: \texttt{3210105252@zju.edu.cn}}}
\affil{(Information and Computing Science 2101), Zhejiang University }


\date{Due time: \today}

\maketitle

\section{Implement the bisection method, Newton’s method, and the secant method in a C++ package.}
\begin{lstlisting}
#include <iostream>
#include <cmath>
#include <functional>
#include <tuple>  

// abstract base class: EquationSolver
class EquationSolver {
public:
    virtual double solve(double initialGuess = 0) = 0;
};

// bisection method
class BisectionSolver : public EquationSolver {
private:
    double a, b, tolerance;
    std::function<double(double)> func;

public:
    BisectionSolver(double a, double b, double tol, std::function<double(double)> f)
        : a(a), b(b), tolerance(tol), func(f) {}

    double solve(double initialGuess = 0) override {
        double mid;
        while ((b - a) / 2 > tolerance) {
            mid = (a + b) / 2;
            if (func(mid) == 0.0)
                return mid;
            else if (func(a) * func(mid) < 0)
                b = mid;
            else
                a = mid;
        }
        return (a + b) / 2;
    }
};

// Newton’s method
class NewtonSolver : public EquationSolver {
private:
    double tolerance;
    std::function<double(double)> func;
    std::function<double(double)> derivative;

public:
    NewtonSolver(double tol, std::function<double(double)> f, std::function<double(double)> df)
        : tolerance(tol), func(f), derivative(df) {}

    double solve(double initialGuess) override {
        double x = initialGuess;
        double h = func(x) / derivative(x);
        while (std::fabs(h) > tolerance) {
            h = func(x) / derivative(x);
            x = x - h;
        }
        return x;
    }
};

// the secant method
class SecantSolver : public EquationSolver {
private:
    double tolerance;
    std::function<double(double)> func;

public:
    SecantSolver(double tol, std::function<double(double)> f)
        : tolerance(tol), func(f) {}

    double solve(double x0, double x1) {
        double x2;
        while (std::fabs(x1 - x0) > tolerance) {
            x2 = x1 - func(x1) * (x1 - x0) / (func(x1) - func(x0));
            x0 = x1;
            x1 = x2;
        }
        return x1;
    }

    double solve(double initialGuess = 0) override {
        return solve(initialGuess, initialGuess + 1.0);
    }
};
\end{lstlisting}
\subsection*{Bisection method}
\subsubsection*{Principle:} The bisection method approaches the root by continuously narrowing the interval containing the solution. We define an interval \([a, b]\), and decide which sub-interval to narrow down next by checking the function value at the midpoint \(\text{mid}\).

\subsubsection*{Design concept:}
\begin{itemize}
    \item \textbf{Constructor Parameters:} \(a\) and \(b\) are the left and right endpoints of the initial interval. \(\text{tolerance}\) is the precision control parameter, and \(\text{func}\) is the function to be solved.
    \item \textbf{Iteration Process:} In each iteration, the interval is halved until the interval width is smaller than \(\text{tolerance}\). At this point, the midpoint is returned as the approximate root.
\end{itemize}
\subsection*{Newton's method}
\subsubsection*{Principle:} Newton's method approaches the root by using the derivative of the function to iteratively approximate the solution. The update formula is \(x_{\text{new}} = x_{\text{old}} - \frac{f(x)}{f'(x)}\), which updates \(x\) using the ratio of the function value to its derivative.

\subsubsection*{Design concept:}
\begin{itemize}
    \item \textbf{Constructor Parameters:} The function \( \text{func} \) and its derivative \( \text{derivative} \), along with the precision control parameter \( \text{tolerance} \), need to be provided.
    \item \textbf{Iteration Process:} By calculating \( \text{func}(x) \) and \( \text{derivative}(x) \), \(x\) is iteratively updated until the absolute value of \( \frac{f(x)}{f'(x)} \) is less than \( \text{tolerance} \).
\end{itemize}
\subsection*{The secant method}
\subsubsection*{Principle:} The secant method does not require the derivative of the function. It uses a difference to approximate the derivative and progressively approaches the root with two initial values \( x_0 \) and \( x_1 \).
\subsubsection*{Design Details:}
\begin{itemize}
    \item \textbf{Iteration Process:} In each iteration, \( x_0 \) and \( x_1 \) are updated using the formula:
    \[
    x_2 = x_1 - f(x_1) \times \frac{x_1 - x_0}{f(x_1) - f(x_0)}
    \]
    until \( |x_1 - x_0| \) is smaller than the specified tolerance.
\end{itemize}

\section{Test your implementation of the bisection method on the
following functions and intervals}
\subsection{\( x^{-1} - \tan x \) on \(\left[ 0, \frac{\pi}{2} \right]\)}
\begin{lstlisting}
auto func1 = [](double x) { return 1.0 / x - std::tan(x); };
BisectionSolver bisect1(0.1, M_PI / 2 - 0.1, 0.0001, func1);
std::cout << "Root of 1/x - tan(x) in [0, pi/2]: " << bisect1.solve() << std::endl;
\end{lstlisting}
\subsubsection*{result:}
Root of $\frac{1}{x} - \tan(x)$ in $[0, \frac{\pi}{2}]$: 0.86028
\subsection{\( x^{-1} - 2^x \) on \([0, 1]\)}
\begin{lstlisting}
auto func2 = [](double x) { return 1.0 / x - std::pow(2, x); };
BisectionSolver bisect2(0.1, 1.0, 0.0001, func2);
std::cout << "Root of 1/x - 2^x in [0, 1]: " << bisect2.solve() << std::endl;
\end{lstlisting}
\subsubsection*{result:}
Root of $\frac{1}{x} - 2^x$ in $[0, 1]$: 0.641132
\subsection{\( 2^{-x} + e^x + 2 \cos x - 6 \) on \([1, 3]\)}
\begin{lstlisting}
auto func3 = [](double x) { return std::pow(2, -x) + std::exp(x) + 2 * std::cos(x) - 6; };
BisectionSolver bisect3(1.0, 3.0, 0.0001, func3);
std::cout << "Root of 2^(-x) + e^x + 2cos(x) - 6 in [1, 3]: " << bisect3.solve() << std::endl;
\end{lstlisting}
\subsubsection*{result:}
Root of $2^{-x} + e^x + 2\cos(x) - 6$ in $[1, 3]$: 1.82941
\subsection{\( \frac{x^3 + 4x^2 + 3x + 5}{2x^3 - 9x^2 + 18x - 2} \) on \([0, 4]\)}
\begin{lstlisting}
auto func4 = [](double x) { return (x * x * x + 4 * x * x + 3 * x + 5) / (2 * x * x * x - 9 * x * x + 18 * x - 2); };
BisectionSolver bisect4(0.1, 4.0, 0.0001, func4);
std::cout << "Root of rational function in [0, 4]: " << bisect4.solve() << std::endl;    
\end{lstlisting}
\subsubsection*{result:}
Root of rational function in $[0, 4]$: 0.117912
\section{Test your implementation of Newton’s method by solving x = tan x. Find the roots near 4.5 and 7.7.}
\begin{lstlisting}
auto func_newton_tan = [](double x) { return x - std::tan(x); };
auto derivative_newton_tan = [](double x) { return 1 - std::pow(1.0 / std::cos(x), 2); };

NewtonSolver newton1(0.0001, func_newton_tan, derivative_newton_tan);
std::cout << "Newton's Method root near 4.5: " << newton1.solve(4.5) << std::endl;

NewtonSolver newton2(0.0001, func_newton_tan, derivative_newton_tan);
std::cout << "Newton's Method root near 7.7: " << newton2.solve(7.7) << std::endl;
\end{lstlisting}
\subsection*{result:}
Newton's Method root near 4.5: 4.49341 \\
Newton's Method root near 7.7: 7.72525
\section{Test your implementation of the secant method by the
following functions and initial values.}
\subsection{$\sin\left(\frac{x}{2}\right) - 1 \quad \text{with} \quad x_0 = 0, \, x_1 = \frac{\pi}{2},$}
\begin{lstlisting}
auto func_secant1 = [](double x) { return std::sin(x / 2) - 1; };
SecantSolver secant1(0.0001, func_secant1);

//  0 and pi/2
std::cout << "Secant Method root of sin(x/2) - 1 with initial values 0 and pi/2: "
<< secant1.solve(0, M_PI / 2) << std::endl;

//  0.1 and 1.5
std::cout << "Secant Method root of sin(x/2) - 1 with new initial values 0.1 and 1.5: "
<< secant1.solve(0.1, 1.5) << std::endl;
\end{lstlisting}
\subsubsection*{result:}
Secant Method root of $\sin\left(\frac{x}{2}\right) - 1$ with initial values $0$ and $\frac{\pi}{2}$: 3.14144 \\
Secant Method root of $\sin\left(\frac{x}{2}\right) - 1$ with new initial values $0.1$ and $1.5$: 3.14143
\subsection{$e^x - \tan{x} \quad \text{with} \quad x_0 = 1, \, x_1 = 1.4,$}
\begin{lstlisting}
auto func_secant2 = [](double x) { return std::exp(x) - std::tan(x); };
SecantSolver secant2(0.0001, func_secant2);

//  1 and 1.4
std::cout << "Secant Method root of e^x - tan(x) with initial values 1 and 1.4: "
<< secant2.solve(1, 1.4) << std::endl;

 //  0.5 and 1.3
std::cout << "Secant Method root of e^x - tan(x) with new initial values 0.5 and 1.3: "
<< secant2.solve(0.5, 1.3) << std::endl;
\end{lstlisting}
\subsubsection*{result:}
Secant Method root of $e^x - \tan{x}$ with initial values $1$ and $1.4$: 1.30633 \\
Secant Method root of $e^x - \tan{x}$ with new initial values $0.5$ and $1.3$: 1.30633
\subsection{$x^3 - 12x^2 + 3x + 1 \quad \text{with} \quad x_0 = 0, \, x_1 = -0.5.$}
\begin{lstlisting}
auto func_secant3 = [](double x) { return x * x * x - 12 * x * x + 3 * x + 1; };
SecantSolver secant3(0.0001, func_secant3);

//  0 and -0.5
std::cout << "Secant Method root of x^3 - 12x^2 + 3x + 1 with initial values 0 and -0.5: "
<< secant3.solve(0, -0.5) << std::endl;

//  1 and 2
std::cout << "Secant Method root of x^3 - 12x^2 + 3x + 1 with new initial values 1 and 2: "
<< secant3.solve(1, 2) << std::endl;
\end{lstlisting}
\subsubsection*{result:}
Secant Method root of $x^3 - 12x^2 + 3x + 1$ with initial values $0$ and $-0.5$: -0.188685 \\
Secant Method root of $x^3 - 12x^2 + 3x + 1$ with new initial values $1$ and $2$: 0.451543
\section{ Find thedepth of water in the trough to within 0.01ft by each of
the three implementations in A.}
\begin{lstlisting}
double L = 10.0;
double r = 1.0;
double V = 12.4;

auto volumeFunc = [L, r, V](double h) {
    return L * (0.5 * M_PI * r * r - r * r * std::asin(h / r) - h * std::sqrt(r * r - h * h)) - V;
};
auto volumeFuncDerivative = [L, r](double h) {
    return L * (-r * r / std::sqrt(r * r - h * h) - std::sqrt(r * r - h * h));
};
\end{lstlisting}
\subsection{Bisection method}
\begin{lstlisting}
BisectionSolver bisectVolume(0.0, r, 0.01, volumeFunc);
std::cout << "Bisection Method: " << bisectVolume.solve() << std::endl;
\end{lstlisting}
\subsubsection*{result:}
Bisection Method: 0.164062
\subsection{Newton's method}
\begin{lstlisting}
NewtonSolver newtonVolume(0.01, volumeFunc, volumeFuncDerivative);
std::cout << "Newton's Method: " << newtonVolume.solve(0.5)<< std::endl;
\end{lstlisting}
\subsubsection*{result:}
Newton's Method: 0.166172
\subsection{The secant method}
\begin{lstlisting}
SecantSolver secantVolume(0.01, volumeFunc);
std::cout << "Secant Method: " << secantVolume.solve(0.1,0.9) << std::endl;
\end{lstlisting}
\subsubsection*{result:}
Secant Method: 0.16617
\section{Failure angle issue of all terrain vehicles}
\begin{lstlisting}
double l = 89.0;
double h = 49.0;
double D1 = 55.0;
double D2 = 30.0;
double beta1_deg = 11.5;
double beta1 = degreesToRadians(beta1_deg);

auto calculateCoefficients = [l, h, beta1](double D) {
    double A = l * std::sin(beta1);
    double B = l * std::cos(beta1);
    double C = (h + 0.5 * D) * std::sin(beta1) - 0.5 * D * std::tan(beta1);
    double E = (h + 0.5 * D) * std::cos(beta1) - 0.5 * D;
    return std::make_tuple(A, B, C, E);
};

auto createFunctions = [](double A, double B, double C, double E) {
    auto func = [A, B, C, E](double alpha) {
        return A * std::sin(alpha) * std::cos(alpha) + B * std::pow(std::sin(alpha), 2)
            - C * std::cos(alpha) - E * std::sin(alpha);
        };
    auto derivative = [A, B, C, E](double alpha) {
        return A * (std::pow(std::cos(alpha), 2) - std::pow(std::sin(alpha), 2))+ 
        2 * B * std::sin(alpha) * std::cos(alpha) + C * std::sin(alpha)- E * std::cos(alpha);
        };
    return std::make_tuple(func, derivative);
};
\end{lstlisting}
\subsection*{a}
\begin{lstlisting}
std::tuple<double, double, double, double> coefficients1 = calculateCoefficients(D1);
double A1 = std::get<0>(coefficients1);
double B1 = std::get<1>(coefficients1);
double C1 = std::get<2>(coefficients1);
double E1 = std::get<3>(coefficients1);

std::tuple<std::function<double(double)>, std::function<double(double)
>> functions1 = createFunctions(A1, B1, C1, E1);
std::function<double(double)> func1_car = std::get<0>(functions1);
std::function<double(double)> derivative1_car = std::get<1>(functions1);

NewtonSolver newtonSolver1(0.0001, func1_car, derivative1_car);
double alpha1_rad = newtonSolver1.solve(degreesToRadians(33));
std::cout << "Verified alpha for D = 55 inches: " << radiansToDegrees(alpha1_rad) << " degrees" 
<< std::endl;
\end{lstlisting}
\subsubsection*{result:}
Verified $\alpha$ for $D = 55$ inches: 32.9722 degrees
\subsection*{b}
\begin{lstlisting}
std::tuple<double, double, double, double> coefficients2 = calculateCoefficients(D2);
double A2 = std::get<0>(coefficients2);
double B2 = std::get<1>(coefficients2);
double C2 = std::get<2>(coefficients2);
double E2 = std::get<3>(coefficients2);

std::tuple<std::function<double(double)>, std::function<double(double)
>> functions2 = createFunctions(A2, B2, C2, E2);
std::function<double(double)> func2_car = std::get<0>(functions2);
std::function<double(double)> derivative2_car = std::get<1>(functions2);

NewtonSolver newtonSolver2(0.0001, func2_car, derivative2_car);
double alpha2_rad = newtonSolver2.solve(degreesToRadians(33));
std::cout << "Alpha for D = 30 inches: " << radiansToDegrees(alpha2_rad) << " degrees"
<< std::endl;
\end{lstlisting}
\subsubsection*{result:}
$\alpha$ for $D = 30$ inches : 33.1689 degrees
\subsection*{c}
\begin{lstlisting}
SecantSolver secantSolver(0.0001, func2_car);
double alpha_secant_rad = secantSolver.solve(degreesToRadians(20), degreesToRadians(60));
std::cout << "Alpha: " << radiansToDegrees(alpha_secant_rad) << " degrees" 
<< std::endl;
\end{lstlisting}
\subsubsection*{result:}
$\alpha$: 33.1689 degrees
\end{document}
