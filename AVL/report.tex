\documentclass[a4paper]{article}
\usepackage[affil-it]{authblk}
\usepackage[backend=bibtex,style=numeric]{biblatex}
\usepackage{listings}
\usepackage{indentfirst}
\setlength{\parindent}{2em}
\usepackage{xcolor}
\usepackage{float}
\usepackage{amsfonts}
\usepackage{tikz}
\usepackage{xeCJK}
\usepackage{amsmath}
\usepackage{graphicx}
\usepackage{setspace}
\setlength{\parindent}{2em}
\lstset{
  language= C++,                
  basicstyle=\ttfamily\small,   
  keywordstyle=\color{blue},    
  stringstyle=\color{green},      
  commentstyle=\color{red},   
  numbers=left,                 
  numberstyle=\tiny,           
  stepnumber=1,                 
  tabsize=4,                    
  showspaces=false,             
  showstringspaces=false        
}
\usepackage{geometry}
\geometry{margin=1.5cm, vmargin={0pt,1cm}}
\setlength{\topmargin}{-1cm}
\setlength{\paperheight}{29.7cm}
\setlength{\textheight}{25.3cm}

\addbibresource{citation.bib}

\begin{document}
% =================================================
\title{实现remove函数修改的说明}

\author{张睿涛 ZhangRuitao 3210105252
  \thanks{Electronic address: \texttt{3210105252@zju.edu.cn}}}
\affil{(Information and Computing Science 2101), Zhejiang University }

\date{Due time: \today}
\maketitle
\begin{onehalfspacing}

\section*{说明}
\begin{itemize}
    \item  \textbf{删除思路:}在删除后,确保每个节点的平衡因子在 -1 到 1 。若某个节点的平衡因子超出这个范围,则通过进行旋转操作(左旋或右旋)以恢复平衡。
    \item  \textbf{remove 函数}
    \begin{itemize}
        \item \textbf{查找节点}:首先,根据传入的值 x 在树中查找待删除的节点。
        \item \textbf{节点删除操作}:一旦找到待删除的节点 t,根据 t 的子节点情况进行不同的处理:
        \begin{itemize}
            \item \textbf{若 t 有两个子节点:}找到 t 右子树的最小节点,并将其值复制到 t 中。之后,递归删除 t 右子树中的最小节点(利用detachMin函数)。
            \item \textbf{若 t 只有一个或没有子节点:}若 t 的左子树不为空,则将 t 替换为其左子树;否则,将 t 替换为其右子树。即删除 t,并将它的子树提升到 t 的位置。
        \end{itemize}
    \end{itemize}
    \item \textbf{更新每个节点高度:}删除节点后,更新受影响节点的高度。高度通过 $updateHeight(t)$ 函数更新,该函数计算节点的左右子树高度,并令节点的高度为 $max(\text{左子树高度}, \text{右子树高度}) + 1$。
    \item \textbf{平衡调整:}删除节点后,检查节点的平衡因子。平衡因子由 
    $ balanceFactor(t)$ 计算:\\
    \[
    balanceFactor(t) = height(t->left) - height(t->right)
    \]
    根据平衡因子的值,进行以下操作:
    \begin{itemize}
        \item \textbf{大于 1(左重):}如果左子树的平衡因子小于 0,则进行左旋操作(将当前节点与其左子树交换,使得左子树成为新的根节点),将左子树的右子树向上提,然后再进行右旋操作。否则,直接进行右旋操作,调整当前节点的平衡。
        \item \textbf{小于 -1(右重):}如果右子树的平衡因子大于 0,则进行右旋操作(将当前节点与其右子树交换,使得右子树成为新的根节点),将右子树的左子树向上提,然后再进行左旋操作。否则,直接进行左旋操作,调整当前节点的平衡。
    \end{itemize}
\end{itemize}
\end{onehalfspacing}

\end{document}

