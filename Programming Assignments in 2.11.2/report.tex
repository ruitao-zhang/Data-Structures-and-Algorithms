\documentclass[a4paper]{article}
\usepackage[affil-it]{authblk}
\usepackage[backend=bibtex,style=numeric]{biblatex}
\usepackage{listings}
\usepackage{xcolor}
\usepackage{xeCJK}
\usepackage{float}
\usepackage{indentfirst}
\usepackage{amsmath}
\usepackage{graphicx}
\usepackage{algorithm}
\usepackage{algorithmic}
\lstset{
  language=C++,                
  basicstyle=\ttfamily\small,   
  keywordstyle=\color{blue},    
  stringstyle=\color{green},      
  commentstyle=\color{red},   
  numbers=left,                 
  numberstyle=\tiny,           
  stepnumber=1,                 
  tabsize=4,                    
  showspaces=false,             
  showstringspaces=false        
}
\usepackage{geometry}
\geometry{margin=1.5cm, vmargin={0pt,1cm}}
\setlength{\topmargin}{-1cm}
\setlength{\paperheight}{29.7cm}
\setlength{\textheight}{25.3cm}

\addbibresource{citation.bib}

\begin{document}
% =================================================
\title{Programming Assignments in 2.11.2}

\author{张睿涛 Zhang Ruitao 3210105252
  \thanks{Electronic address: \texttt{3210105252@zju.edu.cn}}}
\affil{(Information and Computing Science 2101), Zhejiang University }


\date{Due time: \today}

\maketitle



\section*{A}
\subsection*{说明}
\noindent 1. $x$: \texttt{const std::vector<double>\& x} \\
\\
这是输入的 $x$ 值的向量,插值节点。$x$ 是包含 $n$ 个不同 $x_i$ 值的向量。\\
\\
2. $y$: \texttt{const std::vector<double>\& y} \\
\\
这是输入的 $y$ 值的向量,代表对应于 $x$ 的函数值 $f(x)$。\\
\\
3. $target$: \texttt{double target}\\
\\
这是目标点 $x$,希望通过插值多项式 $p_n(f; x_0, x_1, \dots, x_n)$ 来估算函数 $f$ 在该点处的值。\\
\\
4. $Diff$: \texttt{std::vector<std::vector<double>> Diff}\\
\\
这是一个二维数组,用于存储差商表格。其大小为 $n \times n$,其中 $\texttt{Diff}[i][j]$ 表示第 $i$ 行,第 $j$ 列的差商值。\\
\\
\subsection*{算法流程}
\subsubsection*{1. 输入}
给定一个有序的点集 $(x_0, x_1, \dots, x_n)$ 和对应的 $f(x_i)$ 值。

\subsubsection*{2. 构建差商表格}
\begin{itemize}
    \item 初始化一个二维数组 \texttt{Diff},其中 \texttt{Diff[i][0]} 表示函数值 $f(x_i)$。
    \item 利用递归公式计算高阶差商:
    \[
    f[x_i, x_{i+1}, \dots, x_{i+j}] = \frac{f[x_{i+1}, \dots, x_{i+j}] - f[x_i, \dots, x_{i+j-1}]}{x_{i+j} - x_i}
    \]
    \item 逐步完善差商表格 \texttt{Diff}。
\end{itemize}

\subsubsection*{3. 计算插值多项式}
\begin{itemize}
    \item 初始化 \texttt{result = Diff[0][0]}。
    \item 然后递归构建多项式:
    \[
    P_n(x) = f[x_0] + f[x_0, x_1](x - x_0) + f[x_0, x_1, x_2](x - x_0)(x - x_1) + \dots
    \]
    \item 逐项累加差商乘以多项式。
\end{itemize}

\subsection*{4. 返回插值结果}
返回 \texttt{result},即插值多项式 $P_n(x)$ 在目标点处的值。

\subsection*{流程图}

\begin{verbatim}
Start
  |
  V
Input: x (插值点), y (函数值), target (目标点)
  |
  V
Initialize: Diff (差商表)
  |
  V
Step 1: 构建分差表 Diff
  - For each i, set Diff[i][0] = y[i]
  - For each j from 1 to n-1:
      For each i from 0 to n-j-1:
        Compute Diff[i][j] using divided difference formula
  |
  V
Step 2: 在目标处计算插值多项式
  - Initialize result = Diff[0][0]
  - Initialize product = 1.0
  - For i = 1 to n-1:
      Update product *= (target - x[i-1])
      Update result += Diff[0][i] * product
  |
  V
Step 3: Return result
  |
  V
End
\end{verbatim}
\section*{B,C}
\subsection*{解答}
对于区间 $[-5, 5]$ 从 $x_{\text{start}}$ 到 $x_{\text{end}}$ 以步长 $0.1$ 遍历每个 $x$。

对于每个 $n$,计算插值节点 $x_i$ 和对应的函数值 $f_i$。

节点 $x_i$ 的选择是等间距划分的节点,从 $x_{\text{start}}$ 到 $x_{\text{end}}$,一共 $n+1$ 个节点。

使用 \texttt{newton} 函数进行牛顿插值,其中 $x_i$ 是插值节点,$f_i$ 是插值节点对应的函数值,$x$ 是需要插值的点。

将当前 $x$、$f(x)$ 以及插值多项式的值 $p_n(x)$ 写入文件。

最后使用文件中的数据利用python绘图。
\begin{figure}[H] 
\centering 
\includegraphics[width=0.6\textwidth]{Figure_1.png}
\end{figure}
\section*{D}
\subsection*{说明}
\begin{itemize}
    \item \texttt{hermite} 函数:用于计算 Hermite 插值多项式的值,给定时间点 \( t \) 输出相应的位移。
    \item \texttt{hermite1} 函数:计算 Hermite 插值多项式的一阶导数,用于预测在给定时间 \( t \) 的速度。
\end{itemize}

\subsection*{算法流程}
\subsubsection*{hermite}
该函数的输入为时间、位移和速度的向量及所需预测的时间点 \( t \)。核心计算是将给定时间点 \( t \) 映射到相邻的时间区间 \([t_0, t_1]\) 内,并使用 Hermite 基函数对该区间的位移和速度进行插值。
\\
\\
\\
\\

\begin{algorithm}
\caption{Hermite 插值函数}
\begin{algorithmic}[1]
\STATE 输入:\texttt{time}(时间),\texttt{displacement}(位移),\texttt{velocity}(速度),\texttt{t}(预测的时间)
\STATE 初始化 \texttt{result} 为 $0$
\FOR{每一个区间 \([t_0, t_1]\)}
    \IF{\( t \in [t_0, t_1] \)}
        \STATE 计算参数 \( u = \frac{t - t_0}{t_1 - t_0} \)
        \STATE 将 \( s_0 \)、\( v_0 \)、\( s_1 \)、\( v_1 \) 分别赋值为区间端点的位移和速度
        \STATE 使用函数计算 Hermite 插值值:\\
        \quad \texttt{result +=} \( s_0 \cdot h_1(u) + v_0 \cdot (t_1 - t_0) \cdot h_2(u) + s_1 \cdot h_3(u) + v_1 \cdot (t_1 - t_0) \cdot h_4(u) \)
        \STATE 跳出循环
    \ENDIF
\ENDFOR
\RETURN \texttt{result}
\end{algorithmic}
\end{algorithm}

\subsubsection*{hermite1}
流程与 \texttt{hermite} 类似,主要区别在于使用导数形式的函数计算 Hermite 插值多项式的导数。

\begin{algorithm}
\caption{Hermite 插值导数函数}
\begin{algorithmic}[t]
\STATE 输入:\texttt{time}(时间向量),\texttt{displacement}(位移向量),\texttt{velocity}(速度向量),\texttt{t}(预测时间)
\STATE 初始化 \texttt{result} 为 $0$
\FOR{每一个区间 \([t_0, t_1]\)}
    \IF{\( t \in [t_0, t_1] \)}
        \STATE 计算参数 \( u = \frac{t - t_0}{t_1 - t_0} \) 和区间长度 \texttt{dt} 为 \( t_1 - t_0 \)
        \STATE 将 \( s_0 \)、\( v_0 \)、\( s_1 \)、\( v_1 \) 分别赋值为区间端点的位移和速度
        \STATE 使用导数函数计算 Hermite 插值导数:\\
        \quad \texttt{result +=} \( s_0 \cdot h_{11}(u, \texttt{dt}) + v_0 \cdot h_{21}(u, \texttt{dt}) \cdot \texttt{dt} + s_1 \cdot h_{31}(u, \texttt{dt}) + v_1 \cdot h_{41}(u, \texttt{dt}) \cdot \texttt{dt} \)
        \STATE 跳出循环
    \ENDIF
\ENDFOR
\RETURN \texttt{result}
\end{algorithmic}
\end{algorithm}

\section*{E}
\subsection*{解答}
即通过 $newton(days, sp1, day)$ 和 $newton(days, sp2, day)$ 计算两个样本在第 43 天的平均重量。通过牛顿插值多项式返回在给定 day 上的重量近似值。
\begin{lstlisting}
int main(){
    std::vector<double> days = {0, 6, 10, 13, 17, 20, 28};
    std::vector<double> sp1 = {6.67, 17.3, 42.7, 37.3, 30.1, 29.3, 28.7};
    std::vector<double> sp2 = {6.67, 16.1, 18.9, 15.0, 10.6, 9.44, 8.89};
    double day = 43;
    double sp_1 = newton(days, sp1, day)/1000;
    double sp_2 = newton(days, sp2, day)/1000;
    if (sp_1 <= 0) {
        std::cout << "sp1: " << sp_1 << "  die" <<std::endl;
    } else {
        std::cout << "sp1: " << sp_1 << "  survive" << std::endl;
    }
    if (sp_2 <= 0) {
        std::cout << "sp2: " << sp_2 << "  die" << std::endl;
    } else {
        std::cout << "sp2: " << sp_2 << "  survive" << std::endl;
    }
\end{lstlisting}
\section*{F}
\subsection*{解答}
1. 生成心形曲线的点

首先,利用给定的心形曲线方程:
\[
x^2 + \left(1.5y - \sqrt{|x|}\right)^2 = 3
\]
生成一组符合此方程的点 \((x, y)\),作为逼近贝塞尔曲线的标记点。定义了函数 \texttt{heart\_curve(n\_points)},它接受一个参数 \(n\_points\),表示生成的点数量。

2. 计算切线

为计算每个标记点处的切线,使用有限差分,函数 \texttt{calculate\_tangents(points)} 基于标记点之间的坐标差值近似切线。

3. 构建三次贝塞尔曲线段

根据 Algorithm 2.74 的公式,每对相邻的标记点 \(p_j\) 和 \(p_{j+1}\) 可以构建一个三次贝塞尔曲线段。对于每个标记点对 \(p_j\) 和 \(p_{j+1}\),贝塞尔曲线段的控制点 \((q_0, q_1, q_2, q_3)\) 计算如下:
\[
q_0 = p_j
\]
\[
q_1 = p_j + \frac{1}{3} \gamma'(p_j)
\]
\[
q_2 = p_{j+1} - \frac{1}{3} \gamma'(p_j)
\]
\[
q_3 = p_{j+1}
\]
其中,\(\gamma'(p_j)\) 表示第 \(p_j\) 点的切线向量。构造函数 \texttt{bezier\_points(p0, p1, tangent)} 使用此公式构造单个贝塞尔曲线段。

4. 贝塞尔曲线的绘制

为了得到三次贝塞尔曲线段在不同参数 \(t\) 值下的坐标,定义函数 \texttt{evaluate(segment, t)}。利用贝塞尔曲线的参数方程,对每个 \(t\) 计算出曲线上的点,并返回这些点的坐标。

5. 绘制心形曲线和贝塞尔逼近
\begin{lstlisting}
    # Heart
def heart_curve(n_points):
    x = np.linspace(-np.sqrt(3), np.sqrt(3), n_points)
    y_1 = (np.sqrt(3 - x**2) + np.sqrt(np.abs(x))) / 1.5
    y_0 = (-np.sqrt(3 - x**2) + np.sqrt(np.abs(x))) / 1.5
    points_1 = np.column_stack((x, y_1))
    points_0 = np.column_stack((x, y_0))
    return np.concatenate((points_1, points_0[::-1]), axis=0)

def Tangents(points):
    tangents = np.gradient(points, axis=0)
    return tangents

def bezier_points(p0, p1, tangent):
    q0 = p0
    q1 = p0 + (1 / 3) * tangent
    q2 = p1 - (1 / 3) * tangent
    q3 = p1
    return np.array([q0, q1, q2, q3])

def bezier_curve(points, tangents):
    segments = []
    for j in range(len(points) - 1):
        segments.append(bezier_points(points[j], points[j+1], tangents[j]))
    return segments

def evaluate(segment, t):
    q0, q1, q2, q3 = segment
    return ((1 - t)**3)[:, None] * q0 + \
           (3 * (1 - t)**2 * t)[:, None] * q1 + \
           (3 * (1 - t) * t**2)[:, None] * q2 + \
           (t**3)[:, None] * q3

def plot_22(m1):
    heart_points = heart_curve(200)

    plt.figure(figsize=(8, 8))
    plt.plot(heart_points[:, 0], heart_points[:, 1], 'k-', label="True Heart")

    for m in m1:
        points = heart_curve(m+1)
        tangents = Tangents(points)

        segments = bezier_curve(points, tangents)

        points_all = np.vstack([
            evaluate(segment, np.linspace(0, 1, 100))
            for segment in segments
        ])

        plt.plot(points_all[:, 0], points_all[:, 1], label=f'Bezier m={m}')

    plt.xlabel('x')
    plt.ylabel('y')
    plt.title("Heart Curve and Bezier Approximations")
    plt.legend()
    plt.grid(True)
    plt.gca().set_aspect('equal', adjustable='box')
    plt.show()

m1 = [10, 40, 160]
plot_22(m1)
\end{lstlisting}

用函数 \texttt{plot\_22(m1)} 绘制。
\begin{figure}[H] 
\centering 
\includegraphics[width=0.6\textwidth]{Figure_3.png}
\end{figure}



\end{document}
