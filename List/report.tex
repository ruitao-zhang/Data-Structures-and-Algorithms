\documentclass[UTF8]{ctexart}
\usepackage{geometry, CJKutf8}
\geometry{margin=1.5cm, vmargin={0pt,1cm}}
\setlength{\topmargin}{-1cm}
\setlength{\paperheight}{29.7cm}
\setlength{\textheight}{25.3cm}
\usepackage[affil-it]{authblk}
\usepackage[backend=bibtex,style=numeric]{biblatex}
\usepackage{listings}
\usepackage{xcolor}
\usepackage{xeCJK}
\usepackage{amsmath}
\lstset{
  language=C++,                
  basicstyle=\ttfamily\small,   
  keywordstyle=\color{blue},    
  stringstyle=\color{green},      
  commentstyle=\color{red},   
  numbers=left,                 
  numberstyle=\tiny,           
  stepnumber=1,                 
  tabsize=4,                    
  showspaces=false,             
  showstringspaces=false        
}
\usepackage{geometry}
\geometry{margin=1.5cm, vmargin={0pt,1cm}}
\setlength{\topmargin}{-1cm}
\setlength{\paperheight}{29.7cm}
\setlength{\textheight}{25.3cm}

\addbibresource{citation.bib}

% useful packages.
\usepackage{amsfonts}
\usepackage{amsmath}
\usepackage{amssymb}
\usepackage{amsthm}
\usepackage{enumerate}
\usepackage{graphicx}
\usepackage{multicol}
\usepackage{fancyhdr}
\usepackage{layout}
\usepackage{listings}
\usepackage{float, caption}

\lstset{
    basicstyle=\ttfamily, basewidth=0.5em
}

% some common command
\newcommand{\dif}{\mathrm{d}}
\newcommand{\avg}[1]{\left\langle #1 \right\rangle}
\newcommand{\difFrac}[2]{\frac{\dif #1}{\dif #2}}
\newcommand{\pdfFrac}[2]{\frac{\partial #1}{\partial #2}}
\newcommand{\OFL}{\mathrm{OFL}}
\newcommand{\UFL}{\mathrm{UFL}}
\newcommand{\fl}{\mathrm{fl}}
\newcommand{\op}{\odot}
\newcommand{\Eabs}{E_{\mathrm{abs}}}
\newcommand{\Erel}{E_{\mathrm{rel}}}


\begin{document}

\title{进一步完善List.h}

\pagestyle{fancy}
\fancyhead{}
\lhead{张睿涛, 3210105252}
\chead{数据结构与算法第四次作业}
\rhead{Oct.16th, 2024}

\section*{测试程序的设计思路及结果}
\section{测试设计}
\subsection{插入元素}
测试首先使用 \texttt{push\_back} 函数向链表中插入元素。该函数将元素追加到链表的末尾。通过一个循环向链表中插入 0, 1, 2, 3, 4。

\subsection{自增和自减操作}
在插入元素后,测试通过迭代器验证自增(\texttt{++})和自减(\texttt{--})操作的正确性。首先测试前向迭代,使用 \texttt{++} 运算符从链表的头部遍历到尾部。接着,测试后向迭代,使用 \texttt{--} 运算符从链表的尾部回到头部。

\subsection{头尾访问}
\texttt{front} 和 \texttt{back} 函数分别返回链表的第一个元素和最后一个元素的引用。通过测试,确保这些函数在插入元素后能正确返回期望的元素。

\subsection{删除元素}
通过 \texttt{pop\_front} 和 \texttt{pop\_back} 函数,测试从链表头部和尾部删除元素的功能,并打印链表内容以验证操作的正确性。

\subsection{删除区间}
测试 \texttt{erase} 函数,使用它删除链表中从第二个元素到尾部的所有元素。

\subsection{清空链表}
测试 \texttt{clear} 函数,确保该函数能删除链表中的所有元素,并且链表的大小正确更新为 0。

\section{测试结果}
\subsection{插入元素}
使用 \texttt{push\_back} 插入元素时,链表正确存储了从 0 到 4 的值。通过迭代器遍历链表时,按顺序输出这些元素,符合预期。

\begin{verbatim}
插入: 0 1 2 3 4
链表插入后的结果: 0 1 2 3 4
\end{verbatim}

\subsection{自增和自减操作}
在迭代器测试中,\texttt{++} 运算符正确地从链表头部遍历到了链表末尾,\texttt{--} 运算符则成功地从链表末尾回溯到链表头部。

\begin{verbatim}
前向迭代结果: 0 1 2 3 4
后向迭代结果: 4 3 2 1 0
\end{verbatim}

\subsection{头尾访问}
\texttt{front} 函数正确返回了链表的第一个元素 0 ,而 \texttt{back} 函数返回了链表的最后一个元素 4 。

\begin{verbatim}
front: 0
back: 4
\end{verbatim}

\subsection{删除元素}
通过 \texttt{pop\_front} 删除链表的第一个元素,并通过 \texttt{pop\_back} 删除链表的最后一个元素后,链表的剩余元素输出正确。

\begin{verbatim}
删除头尾元素后的链表: 1 2 3
\end{verbatim}

\subsection{删除元素}
\texttt{erase} 函数成功删除了链表中从第二个位置到最后一个位置的元素,最终链表中只剩下第一个元素。

\begin{verbatim}
区间删除后的链表: 1
\end{verbatim}

\subsection{清空链表}
\texttt{clear} 函数成功清空了链表,链表大小被正确更新为 0。

\begin{verbatim}
清空链表后的大小: 0
\end{verbatim}

\end{document}
