\documentclass[a4paper]{article}
\usepackage[affil-it]{authblk}
\usepackage[backend=bibtex,style=numeric]{biblatex}
\usepackage{listings}
\usepackage{indentfirst}
\setlength{\parindent}{2em}
\usepackage{xcolor}
\usepackage{float}
\usepackage{amsfonts}
\usepackage{tikz}
\usepackage{xeCJK}
\usepackage{amsmath}
\usepackage{graphicx}
\usepackage{booktabs}
\setlength{\parindent}{2em}
\lstset{
  language= C++,                
  basicstyle=\ttfamily\small,   
  keywordstyle=\color{blue},    
  stringstyle=\color{green},      
  commentstyle=\color{red},   
  numbers=left,                 
  numberstyle=\tiny,           
  stepnumber=1,                 
  tabsize=4,                    
  showspaces=false,             
  showstringspaces=false        
}
\usepackage{geometry}
\geometry{margin=1.5cm, vmargin={0pt,1cm}}
\setlength{\topmargin}{-1cm}
\setlength{\paperheight}{29.7cm}
\setlength{\textheight}{25.3cm}

\addbibresource{citation.bib}

\begin{document}
% =================================================
\title{用STL工具实现堆排序算法}

\author{张睿涛 Zhang Ruitao 3210105252
  \thanks{Electronic address: \texttt{3210105252@zju.edu.cn}}}
\affil{(Information and Computing Science 2101), Zhejiang University }

\date{Due time: \today}
\maketitle

\section*{一. 整体设计思路}

\begin{enumerate}
    \item 基于 \texttt{std::make\_heap} 和 \texttt{std::pop\_heap} 的原理实现手动堆排序;
    \item 生成测试序列:随机序列、有序序列、逆序序列、以及部分重复序列;
    \item 通过 \texttt{std::chrono} 测量手动堆排序与 STL 堆排序的耗时,并进行对比;
    \item 通过自定义检查函数验证排序正确性。
\end{enumerate}

\section*{二. 函数功能与实现细节}

\subsection*{1.堆排序函数(heapSort)}
\textbf{功能}:实现基于堆操作的排序。

\textbf{实现细节}:
\begin{enumerate}
    \item 使用 \texttt{makeheap} 将数组转换为最大堆;
    \item 使用 \texttt{popheap} 将堆顶元素逐个移至序列末尾;
\end{enumerate}

\textbf{实现代码}:
\begin{lstlisting}
void heapSort(vector<int>& vec) {
    makeHeap(vec); 
    for (size_t i = vec.size(); i > 1; --i) {
        popheap(vec, i); 
    }
}
\end{lstlisting}

\subsection*{2.手动堆调整(heapify)}
\textbf{功能}:调整以某个节点为根的子树,使其满足堆的性质:

\textbf{实现细节:}
\begin{enumerate}
    \item 比较当前节点和它的左右子节点,找到最大(或最小)的值。
    \item 如果当前节点不是最大(或最小)的值,与较大的子节点交换位置。
    \item 递归地对交换位置后的子树进行调整,直到整个子树满足堆的性质。
\end{enumerate}

\textbf{实现代码}
\begin{lstlisting}
void heapify(vector<int>& vec, size_t n, int i) {
    int largest = i;
    int left = 2 * i + 1;
    int right = 2 * i + 2;

    if (left < n && vec[left] > vec[largest]) {
        largest = left;
    }

    if (right < n && vec[right] > vec[largest]) {
        largest = right;
    }

    if (largest != i) {
        swap(vec[i], vec[largest]);
        heapify(vec, n, largest);
    }
}
\end{lstlisting}

\subsection*{3.手动堆构建(makeHeap)}
\textbf{功能}:将一个数组转化为堆

\textbf{实现细节}:从最后一个非叶子节点开始(索引为 $\frac{1}{2}n-1$),逐步向上调用 heapify,保证当前节点及其子树满足堆的性质。

\textbf{实现代码}
\begin{lstlisting}
void makeHeap(vector<int>& vec) {
    size_t n = vec.size();
    for (int i = n / 2 - 1; i >= 0; --i) {
        heapify(vec, n, i);
    }
}
\end{lstlisting}

\subsection*{4.手动pop\_heap(popheap)}
\textbf{功能}:将堆顶移动到数组末尾,同时调整剩余部分使其仍然是一个堆。

\textbf{实现细节}:将堆顶元素与当前堆的最后一个元素交换。调整剩余部分(heapSize - 1)为一个堆。

\textbf{实现代码}:

\begin{lstlisting}
void popheap(vector<int>& vec, size_t heapSize) {
    if (heapSize > 1) {
        swap(vec[0], vec[heapSize - 1]);
        heapify(vec, heapSize - 1, 0);
    }
}
\end{lstlisting}

\subsection*{5.STL 堆排序(SortHeap)}
\textbf{功能}:使用标准库的 \texttt{std::sort\_heap} 实现排序。

\textbf{实现代码}:
\begin{lstlisting}
void SortHeap(vector<int>& vec) {
    make_heap(vec.begin(), vec.end());
    sort_heap(vec.begin(), vec.end());
}
\end{lstlisting}

\subsection*{6.检查函数 (check)}
\textbf{功能}:验证排序结果是否按升序排列。

\textbf{实现代码}:
\begin{lstlisting}
bool check(const vector<int>& vec) {
    for (size_t i = 1; i < vec.size(); ++i) {
        if (vec[i - 1] > vec[i]) return false;
    }
    return true;
}
\end{lstlisting}

\subsection*{7.测试数据生成函数}
生成方式:
\begin{enumerate}
    \item 随机序列:使用均匀分布生成;
    \item 有序序列:从小到大递增;
    \item 逆序序列:从大到小递减;
    \item 部分重复序列:生成指定数量的唯一值。
\end{enumerate}

\textbf{生成随机序列的代码}:
\begin{lstlisting}
vector<int> generateRandomSequence(size_t n, int minVal = 0, int maxVal = 1000000) {
    vector<int> vec(n);
    random_device rd;
    mt19937 gen(rd());
    uniform_int_distribution<int> dist(minVal, maxVal);
    for (auto& val : vec) {
        val = dist(gen);
    }
    return vec;
}
\end{lstlisting}

\section*{三. 测试流程}

\begin{enumerate}
    \item 生成长度为 1,000,000 的随机序列、有序序列、逆序序列、部分重复序列;
    \item 对每种序列分别进行手动堆排序与 \texttt{std::sort\_heap} 排序;
    \item 通过 \texttt{check} 函数验证排序结果;
    \item 记录两种排序方法的耗时并进行比较大小。
\end{enumerate}

\section*{四. 测试结果}

\noindent \textbf{测试数据}:长度为 1,000,000 的序列。

\begin{verbatim}
测试: 随机序列
手动用时: 491 ms
STL sort_heap time: 524 ms
比较: 手动的比sort_heap快33 ms

测试: 有序序列
手动用时: 328 ms
STL sort_heap time: 372 ms
比较: 手动的比sort_heap快44 ms

测试: 逆序序列
手动用时: 328 ms
STL sort_heap time: 393 ms
比较: 手动的比sort_heap快65 ms

测试: 部分元素重复序列
手动用时: 418 ms
STL sort_heap time: 479 ms
比较: 手动的比sort_heap快61 ms
\end{verbatim}

\textbf{表格}:
\begin{table}[h!]
\centering
\begin{tabular}{@{}lccc@{}}
\toprule
测试类型         & 手动用时 (ms) & sort\_heap 用时 (ms) & 比较 \\ \midrule
随机序列         & 491           & 524                      & 手动的比 sort\_heap 快 33 ms \\
有序序列         & 328           & 372                      & 手动的比 sort\_heap 快 44 ms \\
逆序序列         & 328           & 393                      & 手动的比 sort\_heap 快 65 ms \\
部分元素重复序列 & 418           & 479                      & 手动的比 sort\_heap 快 61 ms \\ \bottomrule
\end{tabular}
\caption{手动排序与 STL sort\_heap 用时对比}
\label{tab:performance_comparison}
\end{table}

\textbf{无报错}:经check函数判断,最终结果没有出现“手动失败”和“STL失败”,故结果正确。
\end{document}
