\documentclass[a4paper]{article}
\usepackage[affil-it]{authblk}
\usepackage[backend=bibtex,style=numeric]{biblatex}
\usepackage{listings}
\usepackage{indentfirst}
\setlength{\parindent}{2em}
\usepackage{xcolor}
\usepackage{xeCJK}
\usepackage{amsmath}
\usepackage{graphicx}
\setlength{\parindent}{2em}
\lstset{
  language= C++,                
  basicstyle=\ttfamily\small,   
  keywordstyle=\color{blue},    
  stringstyle=\color{green},      
  commentstyle=\color{red},   
  numbers=left,                 
  numberstyle=\tiny,           
  stepnumber=1,                 
  tabsize=4,                    
  showspaces=false,             
  showstringspaces=false        
}
\usepackage{geometry}
\geometry{margin=1.5cm, vmargin={0pt,1cm}}
\setlength{\topmargin}{-1cm}
\setlength{\paperheight}{29.7cm}
\setlength{\textheight}{25.3cm}

\addbibresource{citation.bib}

\begin{document}
% =================================================
\title{BinarySearchTree}

\author{张睿涛 3210105252
  \thanks{Electronic address: \texttt{3210105252@zju.edu.cn}}}
\affil{(Information and Computing Science 2101), Zhejiang University }


\date{Due time: \today}
\maketitle
\section*{remove函数}
\subsection*{1.代码}
\begin{lstlisting}
    void remove(const Comparable &x, BinaryNode *&t)
    {
        if (t == nullptr)
            return; 

        if (x < t->element)
            remove(x, t->left);
        else if (t->element < x)
            remove(x, t->right);
        else if (t->left != nullptr && t->right != nullptr) 
        {
            t->element = detachMin(t->right)->element;
        }
        else
        {
            BinaryNode *oldNode = t;
            t = (t->left != nullptr) ? t->left : t->right;
            delete oldNode;
        }
    }
\end{lstlisting}
\subsection*{2.流程}
\begin{itemize}
    \item \textbf{空}:
    \begin{itemize}
        \item 如果当前节点 $t$ 为 $\text{nullptr}$,则函数直接返回。
    \end{itemize}
    \begin{lstlisting}
        if (t == nullptr)
                return;
    \end{lstlisting}
    \item \textbf{查找元素}:
    \begin{itemize}
        \item 若要删除的元素 $x < t.\text{element}$,则递归在左子树中查找:$remove(x, t.\text{left})$。
        \begin{lstlisting}
        if (x < t->element)
            remove(x, t->left);
        \end{lstlisting}
        \item 若 $x > t.\text{element}$,则在右子树中查找:$remove(x, t.\text{right})$。
        \begin{lstlisting}
        else if (t->element < x)
            remove(x, t->right);
        \end{lstlisting}
    \end{itemize}
    
    \item \textbf{找到节点}:
    \begin{itemize}
        \item 当 $x = t.\text{element}$,根据节点的子节点情况,有三种情况:
        \begin{enumerate}
            \item \textbf{树叶}:将 $t$ 指向 $\text{nullptr}$。
            \item \textbf{有一个子节点}:将 $t$ 指向其唯一子节点($t = t.\text{left}$ 或 $t = t.\text{right}$)。
            \begin{lstlisting}
        BinaryNode *oldNode = t;
        t = (t->left != nullptr) ? t->left : t->right;
        delete oldNode;
            \end{lstlisting}
            \item \textbf{有两个子节点}:调用 $detachMin$ 从右子树中找到并分离最小节点,并用该节点的值替换要删除节点的值。
            \begin{lstlisting}
        else if (t->left != nullptr && t->right != nullptr) {
            t->element = detachMin(t->right)->element;
        }
            \end{lstlisting}
        \end{enumerate}
    \end{itemize}
\end{itemize}

\section*{detachMin函数}
\subsection*{1.代码}
\begin{lstlisting}
BinaryNode *detachMin(BinaryNode *&t)
    {
        if (t == nullptr)
            return nullptr;

        if (t->left == nullptr)
        {
            BinaryNode *minNode = t;
            t = t->right;
            return minNode;
        }
        return detachMin(t->left);
    }
\end{lstlisting}
\subsection*{2.流程}
\begin{itemize}
    \item \textbf{空}:
    \begin{itemize}
        \item 若当前节点 $t$ 为 $\text{nullptr}$,返回 $\text{nullptr}$,表示没有最小节点。
    \end{itemize}
    \begin{lstlisting}
       if (t == nullptr)
            return nullptr;
    \end{lstlisting}
    \item \textbf{查找最小节点}:
    \begin{itemize}
        \item 函数递归遍历当前节点的左子节点,找到最小节点。
    \end{itemize}
    
    \item \textbf{分离节点}:
    \begin{itemize}
        \item 当达到最左侧节点($t.\text{left}$ 为 $\text{nullptr}$)时,存储该节点的指针 $minNode$,并将 $t$ 指向其右子节点:$t = t.\text{right}$。
        \item 返回得到的 $minNode$(最小元素)。
        \begin{lstlisting}
    if (t->left == nullptr)
    {
        BinaryNode *minNode = t;
        t = t->right;
        return minNode;
    }
        \end{lstlisting}
    \end{itemize}
\end{itemize}
\newpage
\section*{测试结果及分析}
\begin{itemize}
    \item 初始二叉搜索树为50 30 70 20 40 60 80,测试函数test执行了下列步骤:
    \begin{itemize}
         \item 删去树叶节点(20)
         \item 删去有一个子节点的节点(30)
         \item 删去有两个子节点的节点(50)
         \item 删去一个不存在的元素(100)
         \item 删到只剩一个元素位置
         \item 删去最后一个元素(60)
         \item 在空树中执行删除
    \end{itemize}
    \item 得到的结果为:
    \begin{itemize}
        \item Remove 20 (leaf): 50 30 70 40 60 80 \\  分析:判断20为树叶位置,直接删去。
        \item Remove 30 (one child): 50 40 70 60 80 \\ 分析:判断30具有一个子节点,把t指向了其子节点40,使40上移至30处,删去30。
        \item Remove 50 (two children): 60 40 70 80 \\ 分析:判断50具有两个子节点,再通过datechMin的判断,找到了50的右子树中的最小元素60,并将50替换成60.
        \item Remove 100(non-existent): 60 40 70 80 \\ 分析:找不到100,则什么都不做。
        \item Remove 40 70 and 80: 60 \\ 分析:过渡步骤
        \item Remove last node (60): Empty tree \\ 分析:剩唯一一个元素时,判断为树叶位置,直接删去。
        \item Remove from empty tree: Empty tree \\ 分析:找不到10,则什么都不做。
    \end{itemize}
\end{itemize}

\end{document}
