\documentclass[a4paper]{article}
\usepackage[affil-it]{authblk}
\usepackage[backend=bibtex,style=numeric]{biblatex}
\usepackage{listings}
\usepackage{indentfirst}
\setlength{\parindent}{2em}
\usepackage{xcolor}
\usepackage{float}
\usepackage{amsfonts}
\usepackage{tikz}
\usepackage{xeCJK}
\usepackage{amsmath}
\usepackage{graphicx}
\usepackage{booktabs}
\setlength{\parindent}{2em}
\lstset{
  language= C++,                
  basicstyle=\ttfamily\small,   
  keywordstyle=\color{blue},    
  stringstyle=\color{red},      
  commentstyle=\color{red},   
  numbers=left,                 
  numberstyle=\tiny,           
  stepnumber=1,                 
  tabsize=4,                    
  showspaces=false,             
  showstringspaces=false        
}
\usepackage{geometry}
\geometry{margin=1.5cm, vmargin={0pt,1cm}}
\setlength{\topmargin}{-1cm}
\setlength{\paperheight}{29.7cm}
\setlength{\textheight}{25.3cm}

\addbibresource{citation.bib}

\begin{document}
% =================================================
\title{四则运算计算器}

\author{张睿涛 Zhang Ruitao 3210105252
  \thanks{Electronic address: \texttt{3210105252@zju.edu.cn}}}
\affil{(Information and Computing Science 2101), Zhejiang University }

\date{Due time: \today}
\maketitle
\section*{一、设计思路}
\subsection*{1.设置运算符优先级}
\begin{lstlisting}
    int set(char op) {
    if (op == '+' || op == '-') return 1;
    if (op == '*' || op == '/') return 2;
    return 0;
}
\end{lstlisting}
\subsection*{2.处理四则运算}
\begin{lstlisting}
double rule(double a, double b, char op) {
    switch (op) {
        case '+': return a + b;
        case '-': return a - b;
        case '*': return a * b;
        case '/': 
            if (b == 0) throw runtime_error("");
            return a / b;
        default: throw runtime_error("");
    }
}
\end{lstlisting}
\subsection*{3.检查合法性}

\begin{enumerate}
    \item 是否为空
    \begin{lstlisting}
    if (a.empty()) {
        throw runtime_error("");
    }
    \end{lstlisting}
    \item 不能以运算符开头(但可以以左括号开头)
    \begin{lstlisting}
    if (a[0] == '+' || a[0] == '-' || a[0] == '*' || a[0] == '/') {
        throw runtime_error("");
    }
    \end{lstlisting}
    \item 能以运算符结尾(但可以以右括号结尾)
    \begin{lstlisting}
    if (a[a.size() - 1] == '+' || a[a.size() - 1] == '-' || 
        a[a.size() - 1] == '*' || a[a.size() - 1] == '/') {
        throw runtime_error("");
    }

    \end{lstlisting}
    \item 括号数目是否匹配及空括号处理
    \begin{enumerate}
        \item 左括号后不能是运算符
        \begin{lstlisting}
    if (i + 1 < a.size() && (a[i + 1] == '+' || a[i + 1] == '-' || a[i + 1] == '*' || 
    a[i + 1] == '/' || a[i+1] == ')')) {
         throw runtime_error("");
    }
        \end{lstlisting}
        \item 右括号前不能是运算符
        \begin{lstlisting}
    if (i > 0 && (a[i - 1] == '+' || a[i - 1] == '-' || 
    a[i - 1] == '*' || a[i - 1] == '/' || a[i+1] == '(')) {
        throw runtime_error("");
    }
        \end{lstlisting}
        \item 左右括号数量是否匹配
        \begin{lstlisting}
    if (left != right) {
        throw runtime_error("");
    }
        \end{lstlisting}    
    \end{enumerate}
    \item 除数为零
    \begin{lstlisting}
    for (size_t i = 0; i < a.size(); ++i) {
        if (a[i] == '/' && i + 1 < a.size() && a[i + 1] == '0') {
            throw runtime_error("");
        }
    }
    \end{lstlisting}
    \item 运算符连续出现
    \begin{lstlisting}
    for (size_t i = 1; i < a.size(); ++i) {
        if ((a[i] == '+' || a[i] == '-' || a[i] == '*' || a[i] == '/') && 
            (a[i - 1] == '+' || a[i - 1] == '-' || a[i - 1] == '*' || a[i - 1] == '/')) {
            throw runtime_error("");
        }
    }
    \end{lstlisting} 
\end{enumerate} 

\subsection*{4.计算表达式}
\begin{enumerate}
    \item 用栈来处理表达式中的操作数和运算符
    \begin{lstlisting}
    stack<double> num;  // 存储操作数
    stack<char> opt;       // 存储运算符
    \end{lstlisting}
    \item 遍历所有字符
    \begin{enumerate}
        \item 忽略空格
        \begin{lstlisting}
    if (isspace(expression[i])) {
        continue;
    }
        \end{lstlisting}
        \item 处理数字,尤其是浮点数
        \begin{lstlisting}
    if (isdigit(a[i]) || a[i] == '.') {
            // 如果有小数点,转换为浮点数
            double m = 0;
            bool k = false;
            double n = 1;

            // 读取数字,直到遇到非数字和非小数点
            while (i < a.size() && (isdigit(a[i]) || a[i] == '.')) {
                //处理多个小数点
                if (a[i] == '.') {
                    if (k) {
                        throw runtime_error("");
                    }
                    k = true;    // 标记小数点
                } else {
                    m = m * 10 + (a[i] - '0');
                    if (k) {
                        n *= 10;  // 若是小数,则调整小数点位置
                    }
                }
                ++i;
            }

            if (k) {
                m /= n;  // 调整小数点后的数字
            }
        \end{lstlisting}
        \item 栈操作1
        \begin{lstlisting}
    else if (a[i] == '(') {
        opt.push(a[i]);        // 左括号,压栈
    } else if (a[i] == ')') {
        while (!opt.empty() && opt.top() != '(') {   // 右括号,计算至左括号
            double num2 = num.top(); num.pop();     // 获取栈顶的第二个操作数
            double num1 = num.top(); num.pop();        // 获取栈顶的第一个操作数
            char op = opt.top(); opt.pop();        // 获取栈顶的运算符
            num.push(rule(num1, num2, op));       // 执行运算并将结果压入操作数栈
        }
        opt.pop();  // 弹出 '('
    } else if (a[i] == '+' || a[i] == '-' || a[i] == '*' || a[i] == '/') {
            // 优先处理栈中的运算符
        while (!opt.empty() && set(opt.top()) >= set(a[i])) {     
        // 取出操作数并执行运算,直到当前运算符优先级大于栈顶运算符
            double num2 = num.top(); num.pop();
            double num1 = num.top(); num.pop();
            char op = opt.top(); opt.pop();
            num.push(rule(num1, num2, op));
        }
            // 当前运算符压栈
        opt.push(a[i]);
    }
}
        \end{lstlisting}
        \item 栈操作2
        \begin{lstlisting}
     // 处理剩下的运算符
    while (!opt.empty()) {
        double num2 = num.top(); num.pop();          
        double num1 = num.top(); num.pop();          
        char op = opt.top(); opt.pop();                    
        num.push(rule(num1, num2, op));
    }

    // 得到最终结果(栈顶)
    return num.top();
        \end{lstlisting}
    \end{enumerate}
\end{enumerate}

\section*{二、测试用例及结果分析}
\renewcommand{\arraystretch}{2}
\begin{center}
\begin{tabular}{|c|c|c|c|}
\hline
测试用例 & 表达式 & 计算结果 & 结果是否正确 \\
\hline
加法(整数) & 1+1 & 2 & YES \\
\hline
加法(小数) & 1.2+4.352 & 5.552 & YES\\
\hline
减法(整数) & 3300-241 & 3059 & YES \\
\hline
减法(小数) & 5.425-3.412 & 2.013 & YES\\
\hline
除法(整数) & 224/30 & 7.46667 & YES \\
\hline
除法(小数) & 2.45/3.56 & 0.688202 & YES\\
\hline
混合计算(整数) & 11+3*5-4/2 & 24 & YES \\
\hline
混合计算(小数) & 3.4+3.21*6.4-8.4/3.5 & 21.544 & YES\\
\hline
单个括号的混合计算(整数) & 11+3*(5-4)/2 & 12.5 & YES \\
\hline
单个括号的混合计算(小数) & 3.4+3.21*(6.4-8.4)/3.5 & 1.56571 & YES\\
\hline
多重括号的混合计算(整数) & (11+3*(5-4))/2 & 7 & YES \\
\hline
多重括号的混合计算(小数) & (3.4+3.21*(6.4-8.4))/3.5 & -0.862857 & YES\\
\hline
非法:运算符处于开头 & +100 & ILLEGAL & YES \\
\hline
非法:运算符处于结尾 & 100- & ILLEGAL & YES \\
\hline
非法:连续运算符 & 1++1 & ILLEGAL & YES \\
\hline
非法:除数为0 & 234/0 & ILLEGAL & YES \\
\hline
非法:括号不匹配 & 1+(1+1 & ILLEGAL & YES\\
\hline
非法:( + 其他运算符 & 1+(*1) & ILLEGAL & YES \\
\hline
非法:其他运算符 + ) & 1+)1 & ILLEGAL & YES\\
\hline
合法:括号处于开头  & (1+3)*3 & 12 & YES\\
\hline
合法:括号处于结尾 & 1+3*(3+2) & 16 & YES \\
\hline
\end{tabular}
\end{center}
\end{document}
